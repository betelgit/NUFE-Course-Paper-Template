\documentclass[a4paper]{article}
\usepackage[UTF8]{ctex}
\usepackage{graphicx,array,setspace,tabularx,tikz}
\usepackage{amsmath,amssymb,amsthm,mathrsfs}
\usepackage{ulem} % 用于下划线
\usepackage{etoolbox}
\usepackage{float,url,booktabs} % 画表

\newcommand{\dd}{\mathrm{d}} % 微分号使用正体:"\dd "

\newtheorem{definition}{定义}[section]
\newtheorem{theorem}{定理}[section]
\newtheorem{proposition}{命题}[section]
\newtheorem{example}{例}[section]
\newtheorem{remark}{注}[section]

%% 中文字体与字号设置
\usepackage[UTF8, scheme=plain]{ctex}  % 使用 ctex 套件
\usepackage{fontspec}
\setCJKmainfont{仿宋}      % 正文字体
\setmainfont{Times New Roman}
\setCJKsansfont{黑体}      % 标题字体(无衬线)
\setCJKfamilyfont{fsong}{仿宋}
\setCJKfamilyfont{hei}{黑体}


%% 页面设置
\usepackage[top=2.5cm, bottom=2.5cm, left=3cm, right=2cm]{geometry}
\usepackage{setspace}
% 设置行距为28磅
\newcommand{\zhengwen}{\CJKfamily{fsong}\fontsize{12pt}{28pt}\selectfont}
\newcommand{\zhaiyao}{\CJKfamily{fsong}\fontsize{10.5pt}{28pt}\selectfont}
\newcommand{\cankaowenxian}{\CJKfamily{fsong}\fontsize{10.5pt}{15.75pt}\selectfont}
\newcommand{\biaoge}{\CJKfamily{fsong}\fontsize{10.5pt}{13.125pt}\selectfont}


%% 页眉设置
\usepackage{fancyhdr}
\pagestyle{fancy}
\fancyhf{}
\fancyhead[R]{\fontsize{10.5pt}{28pt}\selectfont\CJKfamily{fsong} 南京财经大学本科课程论文}
\fancyfoot[C]{\thepage}


%% 标题格式
\usepackage{titlesec}
% 一级标题:四号黑体
\titleformat{\section}
  {\normalfont\CJKfamily{hei}\fontsize{14pt}{28pt}\selectfont}
  {\thesection}{1em}{}
% 二级及以下标题:小四号仿宋
\titleformat{\subsection}
  {\normalfont\CJKfamily{fsong}\fontsize{12pt}{28pt}\selectfont}
  {\thesubsection}{1em}{}
\titleformat{\subsubsection}
  {\normalfont\CJKfamily{fsong}\fontsize{12pt}{28pt}\selectfont}
  {\thesubsubsection}{1em}{}
% 图表名
\usepackage{caption}
\captionsetup[figure]{
  name=图,
  labelsep=space,
  font=normalsize  % 仿宋,五号
}
\captionsetup[table]{
  name=表,
  labelsep=space,
  font=normalsize  % 仿宋,五号
}
\newcommand{\biaocaption}[1]{\caption{\hspace{1em}#1}}
\newcommand{\tucaption}[1]{\caption{\hspace{1em}#1}}

\setlength{\baselineskip}{28pt} % 统一固定行距28磅


%% 正文段落设置
\setlength{\parindent}{2em}
\setlength{\parskip}{0em}

\makeatletter
\newcommand{\Uline}[2][4cm]{\hskip1pt\underline{\hb@xt@ #1{\hss#2\hss}}\hskip3pt}
\makeatother

\begin{document}

%% 封面页

% 封面页参考部分了Weierstrass老师的设计
% 封面倒是可以用word文档直接打印
\begin{titlepage}

\vspace*{\baselineskip} % 顶部空行

\centerline{\bfseries\heiti\zihao{1} 南~京~财~经~大~学 }\vspace*{6mm}
\centerline{\bfseries\heiti\zihao{2} 本科课程论文}\vspace*{6mm}
\centerline{\zihao{3} 2024 -- 2025 学年第二学期}\vspace*{12mm}
\begin{center}
{\songti\zihao{-3}
课程名称: \Uline[6cm]{\songti\zihao{-3} 课程论文}\par\vspace*{3mm}
题\qquad 目: \Uline[6cm]{ \qquad  南财本科生课程论文\LaTeX 模板}\par\vspace*{3mm}
姓\qquad 名: \Uline[6cm]{ Betelgit }\par\vspace*{3mm}
班\qquad 级: \Uline[6cm]{ 金数2201}\par\vspace*{3mm}
学\qquad 号: \Uline[6cm]{ 212022xxxx}
}
\end{center}

\begin{center}
\par\vspace*{6mm}{\zihao{4}\fangsong
\begin{tabularx}{\textwidth}{|>{\centering\arraybackslash}X|>{\centering\arraybackslash}X|>{\centering\arraybackslash}X|>{\centering\arraybackslash}X|>{\centering\arraybackslash}X|>{\centering\arraybackslash}X|}
  \hline

  \begin{minipage}[c][1.5cm][c]{1.5cm}
    \centering 评分 \\ 项目
  \end{minipage} & 
  \begin{minipage}[c][2cm][c]{1.5cm}\centering 项目1\end{minipage} & 
  \begin{minipage}[c][2cm][c]{\hsize}\centering 项目2\end{minipage} & 
  \begin{minipage}[c][2cm][c]{\hsize}\centering 项目3\end{minipage} & 
  \begin{minipage}[c][2cm][c]{\hsize}\centering 项目4\end{minipage} & 
  \begin{minipage}[c][2cm][c]{\hsize}\centering 总分\end{minipage} \\
  \hline
  
  \begin{minipage}[c][1.5cm][c]{1.5cm}\centering 分数\end{minipage} & 
  \begin{minipage}[c][1.5cm][c]{\hsize}\centering \end{minipage} & 
  \begin{minipage}[c][1.5cm][c]{\hsize}\centering \end{minipage} & 
  \begin{minipage}[c][1.5cm][c]{\hsize}\centering \end{minipage} & 
  \begin{minipage}[c][1.5cm][c]{\hsize}\centering \end{minipage} & 
  \begin{minipage}[c][1.5cm][c]{\hsize}\centering \end{minipage} \\
  \hline
  
  \begin{minipage}[c][1.5cm][c]{1.5cm}\centering 得分\end{minipage} & 
  \begin{minipage}[c][1.5cm][c]{\hsize}\centering \end{minipage} & 
  \begin{minipage}[c][1.5cm][c]{\hsize}\centering \end{minipage} & 
  \begin{minipage}[c][1.5cm][c]{\hsize}\centering \end{minipage} & 
  \begin{minipage}[c][1.5cm][c]{\hsize}\centering \end{minipage} & 
  \begin{minipage}[c][1.5cm][c]{\hsize}\centering \end{minipage} \\
  \hline
  
  \begin{minipage}[c][5cm][c]{1.5cm}
    \centering 教\\师\\评\\语
  \end{minipage} & 
  \multicolumn{5}{c|}{
    \begin{minipage}[c][3cm][c]{\dimexpr\textwidth-1.5cm-8\tabcolsep-2\arrayrulewidth\relax}
      \centering
    \end{minipage}
  } \\
  \hline
\end{tabularx}}
\end{center}

\end{titlepage}


%% 标题
\begin{center}
  {\CJKfamily{hei}\fontsize{22pt}{28pt}\selectfont 论文标题示例}
\end{center}

\begin{center}
  {\CJKfamily{hei}\fontsize{16pt}{28pt}\selectfont --- 论文副标题示例}
\end{center}

\vspace{1em}


%% 摘要:五号仿宋
{\zhaiyao

\textbf{摘要:}这是摘要部分,字体为仿宋体五号。这是摘要部分,字体为仿宋体五号。这是摘要部分,字体为仿宋体五号。这是摘要部分,字体为仿宋体五号。
\par\textbf{关键词:}南京\hspace*{2em}财经\hspace*{2em}大学

}

\vspace{2em}


%% 英文标题
\begin{center}
  {\fontsize{16pt}{28pt}\selectfont \textbf{Nanjing University of Finance and Economics}}
\end{center}

\begin{center}
  {\fontsize{14pt}{28pt}\selectfont \textbf{ --- NUFE}}
\end{center}


%% 英文摘要:五号新罗马
{\fontsize{10.5pt}{28pt}\selectfont

\textbf{Abstract:~} Nanjing University of Finance and Economics Nanjing University of Finance and Economics Nanjing University of Finance and Economics 
Nanjing University of Finance and Economics Nanjing University of Finance and Economics

\par\textbf{Keywords:~} N\hspace*{2em}U\hspace*{2em}F\hspace*{2em}E

}

\newpage


%% 正文内容:小四号仿宋
{\zhengwen
\section*{一、引言}

该模板可以让使用者不再花大量的时间在排版和公式的转换上,\\
从而\sout{可以直接使用AI快速生成内容}可以把精力集中在内容上。\par

正文,二级标题及以下,均使用仿宋小四。\par

\subsection*{(1)公式排版}

    \subsubsection*{1.编号公式}
        \begin{equation}
            \int_{\gamma}^{}f(z)\dd z=2\pi i\cdot \mathrm{res}f(a)
        \label{res}
        \end{equation}
公式(\ref{res})是复变函数的留数与复积分的关系
    \subsubsection*{2.无编号公式}
        
        多行多列公式\footnote{如果公式中有文字出现,请用 $\backslash$mathrm\{\} 或者 $\backslash$text\{\} 包含,不然就会变成 $clip$,在公式里看起来比 $\mathrm{clip}$ 丑非常多。}
        \begin{align*}
            C(S, t) &= S \cdot N(d) - K e^{-r(T-t)} \cdot N(d - \sigma \sqrt{T-t})
            \\ \text{其中有:~} d &= \frac{\ln(S/K) + (r + \sigma^2/2)(T-t)}{\sigma \sqrt{T-t}}
        \end{align*}

\section*{二、图片插入}

\begin{itemize}
        \item 矢量图 eps, ps, pdf
        \begin{itemize}
            \item METAPOST, pstricks, pgf $\ldots$
            \item Xfig, Dia, Visio, Inkscape $\ldots$
            \item Matlab / Excel 等保存为 pdf
        \end{itemize}
        \item 标量图 png, jpg, tiff $\ldots$
        \begin{itemize}
            \item 提高清晰度,避免发虚
            \item 应尽量避免使用
        \end{itemize}
\end{itemize}

\begin{figure}[htpb]
    \centering
    \includegraphics[width=8cm]{fig/nufe.png}
    \tucaption{南财大校标}
    \label{校标}
\end{figure}
图\ref{校标}是校标,不过是标量图.


\section*{三、表格}

{\biaoge
\begin{table}[H]
  \centering  %设置居中
  \biaocaption{符号说明}  %表标题
  \label{notion} %设置表的引用标签
  \begin{tabular}{ccccccc} %7个c表示7列, c表示每列居中对齐, 还有l和r可选
  \toprule  %画顶端横线
  {\bf 符号}  & {\bf 描述}\\[0.2cm]
  \midrule  %画中间横线

$\nabla$ & Laplace算子 \\[0.2cm]

$H$ & Hilbert空间 \\[0.2cm]

$\mathbb{R}^n$ & n维Euclid空间 \\[0.2cm]

$\mathscr{X}$ & 度量空间 \\[0.2cm]

$B^*$ & 赋范线性空间 \\[0.2cm]

$B$ & Banach空间 \\[0.2cm]

$C_{[0,1]}$ & [0,1]上的全体连续函数 \\[0.2cm]

\bottomrule  %画底部横线
\end{tabular}
\end{table}
}
表\ref{notion}是符号说明.


}

\vspace{2em}


%% 参考文献:五号仿宋
{\cankaowenxian

\begin{center}
  {\CJKfamily{hei}\fontsize{12pt}{28pt}\selectfont 参考文献}
\end{center}

[1] 刘国钧,陈绍业,王凤翥. 图书馆目录[M]. 北京:高等教育出版社,1957:15–18.

[2] 辛希孟. 信息技术与信息服务国际研讨会论文集:A集[C]. 北京:中国社会科学出版社,1994.

[3] 张筑生. 微分半动力系统的不变集[D]. 北京:北京大学数学系数学研究所,1983.

[4] 冯西桥. 核反应堆压力管道与压力容器的LBB分析[R]. 北京:清华大学核能技术设计研究院,1997.

[5] Powell, W W, Koput, K W, SmithDoerr, L. Interorganizational collaboration and the locus of innovation[J]. Administrative Science Quarterly, 1996, 41(1):116-145.

}

\end{document}